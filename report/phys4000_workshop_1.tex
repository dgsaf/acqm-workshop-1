\documentclass[draft]{article}

% - Style
\usepackage{base}

% - Title
\title{PHYS4000 - Workshop 1}
\author{Tom Ross - 1834 2884}
\date{}

% - Headers
\pagestyle{fancy}
\fancyhf{}
\rhead{\theauthor}
\chead{}
\lhead{\thetitle}
\rfoot{\thepage}
\cfoot{}
\lfoot{}

% - Document
\begin{document}

\section{Introduction}
\label{sec:introduction}

We consider the one-electron Hydrogenic-atom Hamiltonian, which is of the form
\begin{equation*}
  \hat{H}
  \ket{\psi}
  =
  E
  \ket{\psi}
\end{equation*}
where $\hat{H} = \hat{K} + \hat{V}$, with
\begin{equation*}
  \hat{K}
  \ket{\psi}
  =
  \lrsq[\bigg]
  {
    -
    \dfrac{1}{2 r}
    \pdv[2]{r}
    \lr[\big]
    {
      r
      \cdot
    }
    +
    \dfrac{1}{2 r^{2}}
    \hat{L}^{2}
  }
  \ket{\psi}
  \qq{and}
  \hat{V}
  \ket{\psi}
  =
  -
  \dfrac{Z}{r}
  \ket{\psi}
\end{equation*}
and where $\hat{L} = \hat{L}_{x} + \hat{L}_{y} + \hat{L}_{z}$ is the angular
momentum operator, which has eigenstates $\ket{Y_{\ell}^{m}}$ which satisfy
\begin{equation*}
  \hat{L}^{2}
  \ket{Y_{\ell}^{m}}
  =
  \ell
  \lr{\ell + 1}
  \ket{Y_{\ell}^{m}}
  \qq{and}
  \hat{L}_{z}
  \ket{Y_{\ell}^{m}}
  =
  m
  \ket{Y_{\ell}^{m}}
  .
\end{equation*}
We solve this system by the method of basis expansion, where we utilise a basis
of the form, $\mathcal{B} = \lrset{\ket{\phi_{i}}}_{i = 1}^{N}$ which we suppose
to be complete in the limit as $N \to \infty$.
We select the basis functions, represented in coordinate-space, to be of the
form
\begin{equation*}
  \phi_{i}\lr{r, \Omega}
  =
  \dfrac{1}{r}
  \varphi_{k_{i}, \ell_{i}}\lr{r}
  Y_{\ell_{i}}^{m_{i}}\lr{\Omega}
  \qq{for}
  i = 1, \dotsc, N.
\end{equation*}
For elements of this basis, the one-electron Hydrogenic-atom Hamiltonian assumes
the form
\begin{alignat*}{2}
  \hat{H}
  \ket{\phi_{i}}
  {}={}
  &
  \lrsq[\bigg]
  {
    -
    \dfrac{1}{2 r}
    \pdv[2]{r}
    \lr[\big]
    {
      r
      \cdot
    }
    +
    \dfrac{1}{2 r^{2}}
    \hat{L}^{2}
    -
    \dfrac{Z}{r}
  }
  \ket{\phi_{i}}
  \\
  {}={}
  &
  \lrsq[\bigg]
  {
    -
    \dfrac{1}{2 r}
    \pdv[2]{r}
    \lr[\big]
    {
      r
      \cdot
    }
    +
    \dfrac
    {
      \ell_{i}
      \lr{\ell_{i} + 1}
    }
    {
      2
      r^{2}
    }
    -
    \dfrac{Z}{r}
  }
  \ket{\phi_{i}}
  \\
  {}={}
  &
  \lrsq[\bigg]
  {
    -
    \dfrac{1}{2 r}
    \pdv[2]{r}
    +
    \dfrac
    {
      \ell_{i}
      \lr{\ell_{i} + 1}
    }
    {
      2
      r^{3}
    }
    -
    \dfrac{Z}{r^{2}}
  }
  \ket{\varphi_{k_{i}, \ell_{i}}}
  \otimes
  \ket{Y_{\ell_{i}}^{m_{i}}}
\end{alignat*}
thus reducing to operator which acts purely to radial terms.

\section{Laguerre Basis}
\label{sec:laguerre-basis}

% relevant theory and code
We utilise a Laguerre basis for the set of radial functions which, in
coordinate-space, are of the form
\begin{equation*}
  \varphi_{k, \ell}\lr{r}
  =
  N_{k, \ell}
  \lr
  {
    2
    \alpha
    r
  }^{\ell + 1}
  \exponential\lr
  {
    -
    \alpha
    r
  }
  L_{k - 1}^{2 \ell + 1}\lr{2 \alpha r}
\end{equation*}

%% method to calculate normalisation constant without causing overflow errors

%% bounds on k and l

% figures: (4) different combinations of l, alpha, showing first (4) basis
% functions

%% discuss effect of changing l, alpha

%% choose appropriate r_max to show exponential decay

\section{Kinetic Energy Matrix Elements}
\label{sec:kinet-energy-matr}

% derive kinetic energy matrix elements

\subsection{Extension: Overlap Matrix Elements}
\label{sec:extens-overl-matr}

% derive overlap energy matrix elements

\section{Atomic Hydrogen States}
\label{sec:atom-hydr-stat}

% figure: for l = 0 alpha = 1.0, and varying n_basis, the convergence % of the
% eigenvalues to the hydrogen energy spectrum

% figure: for fixed n_basis and varying alpha, the behaviour of the eigenvalue
% spectrum, and the consequences it could have on subsequent scattering
% calculations

% figures: for a satisfactory converged n_basis, a figure each for l = 0, l = 1,
% comparing the first (3) bound states with analytical solutions

\subsection{$\mathrm{He}^{+}$ Ion}
\label{sec:he+-ion}

% can you reproduce its analytical energies and wave functions?

\subsection{Surface Plot in $xz$ Plane}
\label{sec:surface-plot-xz}

% figure: surface plot, for m = 0, in xz plane

\subsection{Numerically Calculating Potential Matrix Elements}
\label{sec:numer-calc-potent}

% evaluate v matrix elements numerically, discuss how to avoid singularity at
% origin

\end{document}